\documentclass[a4paper,12pt]{article}
\usepackage{fancyhdr,lastpage}
\usepackage{fullpage}
\usepackage{hyperref}
\usepackage{ulem}


\usepackage{natbib}
\usepackage{wasysym}      % \permil
\usepackage{graphicx}
\usepackage{booktabs}
\bibpunct{(}{)}{;}{a}{,}{;}


\pagestyle{fancy}
\lhead{Dynamics 2\\   \quad \\Due date: 19.05.2014}
\rhead{Exercise 2, SS 2014\\Prof. Dr. G. Lohmann\\ Paul Gierz}


%\bibpunct{(}{)}{;}{a}{,}{;}

%often used special strings in normal Text
\def\delO{$\delta^{18}$O}           % delta-18-O
\def\delD{$\delta$D}                % delta-D
\def\H2Oi{H$_2$$^{18}$O}            % H2-18-O
\def\ra{$\Rightarrow$}              % =>
\def\la{$\Leftarrow$}               % <=
\def\deg{$^{\circ}$}                % Grad
\def\degC{$^{\circ}$C}              % Grad Celsius

%often used special strings in formula environment
\def\fdelO{\delta^{18}\textrm{O}}   % delta-18-O
\def\fdelD{\delta\textrm{D}}        % delta-D
\def\fra{\Rightarrow}               % =>
\def\fla{\Leftarrow}                % <=


%often used structures
\def\be{\begin{equation}}
\def\ee{\end{equation}}

\newcommand{\beq}{\begin{eqnarray*}}
\newcommand{\eeq}{\end{eqnarray*}}
\newcommand{\beqn}{\begin{eqnarray}}
\newcommand{\eeqn}{\end{eqnarray}}
\newcommand{\Der}[2]{\frac{\partial #1}{\partial #2}}
\newcommand{\Dtwo}[2]{\frac{\partial^2 #1}{\partial #2^2}}
\newcommand{\Dx}[1]{\frac{\partial #1}{\partial x}}
\newcommand{\Dt}[1]{\frac{\partial #1}{\partial t}}
%\newcommand{\itt}{\footnotesize}
\newcommand{\sff}{\small \sf}
\newcommand{\itt}{\it }
\newcommand{\itc}{\it }
\newcommand{\dx}{\partial_x}
\newcommand{\dy}{\partial_y}
\newcommand{\dz}{\partial_z}
\newcommand{\dt}{\partial_t}
\newcommand{\ad}{{\footnotemark[2]}}


\newcommand{\Eta}{ \boldmath E }
\newcommand{\vbf}{ \boldsymbol }
\newcommand{\fig}{Fig.\,}
\newcommand{\figs}{Figs.\,}
\renewcommand{\beq}{\begin{eqnarray*}}
\renewcommand{\eeq}{\end{eqnarray*}}
\renewcommand{\beqn}{\begin{eqnarray}}
\renewcommand{\eeqn}{\end{eqnarray}}
\newcommand{\OneMUskip}{\mskip 1mu \relax}  
% 18mu=1em f"ur \textfont
\newcommand{\Nat}{{\sf N\OneMUskip}}
\newcommand{\Reals}{{\sf R\OneMUskip}}
\newcommand{\Complex}{{\sf C\OneMUskip}}

%opening
\title{Exercise 2}
\date{\today}

\begin{document}
\maketitle
\thispagestyle{fancy}
\begin{enumerate}
    

\item {\bf Several questions about the course:} (\textit{0.5 points each})\\
\begin{enumerate}
\item The
      Coriolis parameter f is defined as \\
a) $ f =
      \Omega \cos \varphi $\\
b) $ f = 2
      \Omega \cos \varphi $ \\
c) $ f = 2
      \Omega \sin \varphi $ \\
d) $ f =
      \beta y $ \\

\item Please clarify: On the Northern Hemisphere, particles tend to go to the right or left relative to the direction of motion due to the Coriolis force? 

\item Please write down the equation of state for the ocean and atmosphere!

\item What is the hydrostatic approximation in the momentum equations?

\end{enumerate}
\item 
{\bf Short programming questions.}\textit{ (\textit{1 point each})}

 Write down and explain the output for the following R-commands:
 \begin{enumerate}
\item
      \begin{verbatim}0:10 \end{verbatim}
\item
      \begin{verbatim}a<-c(0,5,3,4); mean(a)\end{verbatim} 
\item
      \begin{verbatim}max(a)-min(a) \end{verbatim} 
\item
      \begin{verbatim}paste("The mean value of a is",mean(a),"for
      sure",sep="_") \end{verbatim}
\item
      \begin{verbatim}a*2+c(1,1,1,0)\end{verbatim} 
\item
      \begin{verbatim}
my.fun<-function(n){return(n*n+1)}
my.fun(10)-my.fun(1)\end{verbatim}

\end{enumerate}
\vspace{1cm}
  \item \textbf{Evaluate the double vector product:} (\textit{2 points})
$$ \Omega \times ( \Omega \times r ) $$
with vectors $\Omega=(0,0,\omega)$ and $r=(x,y,z)$.

  \item \textbf{Given $f(x,y,z,t)$. What is the definition of partial derivatives for this variable?
What is the definition of nabla, Laplacean, divergence, total (substantial) derivative, total differential?} (\textit{2 points})

\item {\bf Population Dynamics} (1 point each)\\
 Consider population dynamics with population $x > 0 $ and reproduction (birth-death) $ r $:
      \begin{equation}
          \frac{dx}{dt} = x \cdot r(x) \label{popDyn}
      \end{equation}
      \begin{enumerate}
 \item
      Solve the differential equation for $r(x)=r_0=const.$!\\
      What happens for $t \to \infty $ when $r_0 > 0$ or $r_0 < 0$ ?\\
 \item
      Solve the differential equation for $r=r_0 (1-x) $! (limited
      growth)\\
 What
      happens for $t \to \infty $?\\
 \item
      Consider the case $r=r_0 (1-x/K) $ with $K>0$ ! Give a physical
      interpretation for $ K $! \\
 % \item
      Calculate the bifurcation with respect to parameter $r$;
      illustrate with a bifurcation diagram.

      \end{enumerate}

 \item
      {\bf Difference equations} (\textit{1 point each})\\
 Consider the discretised form of (\ref{popDyn}) using the Euler scheme
 \begin{equation}
      \frac{dx}{dt} \approx \frac{x_{n+1}-x_n}{\Delta t}.
 \end{equation}

      \begin{enumerate}
 \item
      Write down the iteration $x_{n+1}$ as a function of $x_{n}$ .\\
 \item
      What is the solution of $x_{n+1}$ as a function of $x_{0}$?\\
 Consider
      the stability for the cases $r>0$, \, $0 > \Delta t \, r
      > -1$, \, $ -1 > \Delta t \, r > -2$, \, $ -2 > \Delta
      t \, r$ \, . \\
 Do you
      have a graphical interpretation of the oscillation/decay?\\

      \end{enumerate}

\item \textbf{Reynolds Number} (\textit{3 points}) \\
Let us consider a flow problem where the fluid behaves according to the incompressible Navier-Stokes equations: \\
\begin{center}
\begin{align}
\left\{
\begin{array}{r}
      \displaystyle \frac{\partial \vec{u}}{\partial t} +
      \left(\vec{u}\cdot \nabla\right)\vec{u} = -\frac{1}{\rho_0}\nabla
      p + \nu \nabla^2 \vec{u}\\
      \displaystyle \nabla \cdot \vec{u} = 0
\end{array}
\right 
\end{align}
\end{center}
where:
\begin{itemize}
 \item
      $\vec{u}$ = flow velocity;
 \item $t$
      = time;
 \item
      $\rho_0$ = fluid density (constant);
 \item $p$
      = pressure;
 \item
      $\nu$ = kinematic viscosity (fluid property).
\end{itemize}

Using the reference density $\rho_0$ and further assuming that a reference lengthscale $l_0$ and a reference flow velocity $U_0$ have already been defined for the flow, transform these equations into a non-dimensional frame of reference. On how many parameters does the flow depend in the non-dimensional system? What is the role of the Reynolds number ($Re\equiv \frac{U_0 l_0}{\nu}$)?

{\small
      \textit{Hint: Use $U_0,\ \rho_0,\ l_0$ to construct reference values
      for all other involved quantities ($t,p$) and operators
      ($\partial/\partial_t,\ \nabla$), insert them into the equations
      and reduce as much as possible the terms which have constant
      factors.}}

\end{enumerate}


%\hspace{2cm}
\vfill
%\hline
\underline{Notes on submission of the exercises:}
 \textit{
Working in study groups  is encouraged, but each student is responsible for his/her own solution.
Please email your own answers to the questions to Paul.Gierz@awi.de.
}

\end{document}