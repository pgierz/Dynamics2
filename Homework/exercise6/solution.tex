
\documentclass[a4paper,12pt]{article}
\usepackage{fancyhdr,lastpage}
\usepackage{fullpage}
\usepackage{hyperref}
\usepackage{ulem}
\pagestyle{fancy}
\lhead{Dynamics 2\\Lecturer: Prof. Dr. G. Lohmann\\Due date: 23.06.2014}
\rhead{Exercise 6, Summer semester 2014\\Paul Gierz\\16.06.2014}

%opening
\title{Exercise 6 Solution}
%\author{ Gerrit Lohmann}


\begin{document}

\maketitle
\thispagestyle{fancy}

A negative salt perturbation will lead to a decrease in the
overturning, causing less heat transport from the equatorial latitudes
to the high latitudes. Therefore, we expect a cooling as the deep
circulation breaks down, a possible relative warming if the
circulation overshoots its initial state, and an eventual recovery to
the initial state. These sort of cooling events typically last between
50-100 years, depending on the strength of the perturbation, and are
often seen in geological records (so-called Heinrich events).

The reason behind these cooling eventus during a circulation breakdown
is the high heat capacity of water compared to air. Since the majority
of heat is transported via the ocean, not the atmosphere, any change
in global ocean circulation patterns will also lead to a redistrubtion
of temperature patterns. 


\vfill
\underline{Notes on submission form of the exercises:}
 \textit{Students can work together, but each is required to submit
   his or her own solutions. The answers to the questions shall be send to paul.gierz@awi.de.}

\end{document}

