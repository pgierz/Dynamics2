
\documentclass[a4paper,12pt]{article}
\usepackage{fancyhdr,lastpage}
\usepackage{fullpage}
\usepackage{hyperref}
\usepackage{ulem}
\usepackage{amsmath}
\pagestyle{fancy}
\lhead{Dynamics 2\\Lecturer: Prof. Dr. G. Lohmann\\Due date: 14.07.2014}
\rhead{Exercise 8, Summer semester 2014\\Paul Gierz\\07.07.2014}

%opening
\title{Exercise 8: Last Exercise!!}
%\author{ Gerrit Lohmann}
\date{\today}

\begin{document}

\maketitle
\thispagestyle{fancy}
\begin{enumerate}
\item {\bf Ice Sheet Melting} In the present climate the volume of freshwater trapped in ice sheets
over land is ∼ $33 \times 10^{6}$ km$^{3}$. If all this ice melted and ran into
the ocean, estimate by how much the sea level would rise. What would
happen to sea level if all the sea-ice melted? 

\item \textbf{Solution} We know the volume of ice from the problem,
  the ocean area is 71\%, and the earth radius is 6371 km. Therefore,
  we can calculate the amount of sea level rise:

\begin{flalign}
h &= \frac{V_ice}{A_ocean} \\
&= \frac{33 \times 10^6}{362.145 \times 10^6} \\
&= 91.1
\end{flalign}

Since sea ice floats on the ocean surface, it displaces just as much
volume of water as it would generate when melting. Therefore, sea
level rise can only be caused by land ice melting, along with
accompanying second order effects (eg thermal expansion of the ocean)

\item {\bf Wind-driven ocean circulation} When the windstress is only zonal, the Sverdrup transport is:
 \begin{equation}
\rho_0 \beta V = \mbox{curl } \tau = - \frac{\partial}{\partial y} {\tau^x} 
\end{equation}
 and Ekman transports and Ekman pumping velocity are:
\begin{equation}
\rho_0 f
      V_E = -{\tau^x} \\
\rho_0 w_E= \mbox{curl } \tau = - \frac{\partial}{\partial y}  {\tau^x} .
\end{equation}
Assume furthermore
\begin{equation}
\tau^x =
      -\tau_0 \cos (\pi y/B)
\end{equation}
 for an ocean basin $0<x<L, 0<y<B$.
\begin{enumerate}
\item At what latitudes $y$ are $|V|$ and $|V_E|$ maximum? Calculate their
magnitudes. Take constant $ f=10^{-4}\ \rm s^{-1}$ and $\beta=1.8
\cdot 10^{-11}\ \rm m^{-1} s^{-1}$ and $B=5000\ {\rm km},
\tau_0/\rho_0 = 10^{-4}\ \rm m^2 s^{-2}$.
\item Calculate the maximum of $w_E$ for constant $f$ (value see
      above).  Is this measurable?\\
\end{enumerate}

\item {\bf Stochastic Climate Model} Imagine that the temperature of
  the ocean mixed layer of depth h is governed by 
\begin{equation}
      \frac{dT}{dt} = -\lambda T + \frac{Q_{net}}{\gamma_{O}} \, ,
\end{equation}
where coefficient $ \gamma_{O} $ is given by the heat capacity $c_p
\rho h $, and $ \lambda $ is the typical damping rate of a temperature
anomaly.
The air-sea fluxes due to weather systems are represented by a
white-noise process $Q_{net}=\hat{Q}_{\omega}e^{i\omega t}$ where
$\hat{Q}_{\omega}$ is the amplitude of the random forcing at frequency $\omega$. $\hat{Q}^{*}$ is the complex conjugate.

\begin{enumerate}
\item  What is a white-noise process? Remember that 
\begin{equation}
\int_R \exp(i \omega t) \delta(t-0) dt = 1
\end{equation}
and use the Fourier transformation.

\item Solve the equation (4) above for the temperature response $T= \hat{T}_{\omega}
      e^{i\omega t}$ and hence show that:
 \begin{equation}
      \hat{T}_{\omega} = \frac{\hat{Q}_{\omega}}{\gamma_{O}\left({\lambda}+i \omega\right)}
\end{equation}

\item Show that it has a spectral density $\hat{T}_{\omega}\hat{T}^{*}_{\omega}$ is given by:
 \begin{equation}
      \hat{T}\hat{T}^{*}= \frac{\hat{Q}
      \hat{Q}^{*}}{\gamma^{2}_{O}\left({\lambda}^{2}+ \omega^{2}\right)}
\end{equation}
and the spectrum
\begin{equation}
 S(\omega)
      = <\hat{T}\hat{T}^{*}> =
      \frac{1}{\gamma^{2}_{O}\left({\lambda}^{2}+ \omega^{2}\right)}.
\end{equation}
The brackets $ < \dots >$ denote the ensemble mean.
Make a sketch of the spectrum using a log-log plot and show that fluctuations with a frequency greater than ${\lambda}$ are damped.
\end{enumerate}

\item {\bf Angular Momentum and Hadley Cell} Consider a
      zonally symmetric circulation (i.e., one with no longitudinal
      variations) in the atmosphere. In the inviscid upper troposphere
      one expects such a flow to conserve absolute angular momentum,
      i.e., 
\begin{equation}
\frac{DA}{Dt}=0
\end{equation}

where A is the absolute angular momentum per unit mass (parallel to
the Earth's rotation axis)
 \begin{equation}
 A = r
      \left(u + \Omega r \right) = \Omega R^{2} \cos^{2} \varphi + u R
      \cos \varphi
\end{equation}

 $\Omega$ is the Earth rotation rate, $u$ the eastward wind component,
 $ r = R \cos \varphi $ is the distance from the rotation axis, $R$
 the Earth's radius, and $\varphi$ latitude.

\begin{enumerate}

\item Show for inviscid zonally symmetric flow that the relation
  $\frac{DA}{Dt}=0$ is consistent with the zonal component of the equation of motion 
 \begin{equation}
      \frac{Du}{Dt} - f v = -\frac{1}{\rho}\frac{\partial p}{\partial x}
      \label{udyn}
\end{equation}
in $\left(x,y,z\right)$ coordinates, where $y=R\varphi$.

\item Use angular momentum conservation to describe in words how the
  existence of the Hadley circulation explains the existence of both
  the subtropical jet in the upper troposphere and the near-surface trade winds.

\item If the Hadley circulation is symmetric about the equator, and
  its edge is at $20^{\circ}$ latitude, determine the strength of the subtropical jet. Use (10, 11).

\item Is the Hadley cell geostrophically driven or not?
\end{enumerate}

\end{enumerate}


\vfill
\underline{Notes on submission form of the exercises:}
 \textit{Students can work together in groups, but each student must
   submit his or her own solutions. The answers to the questions shall be send to paul.gierz@awi.de.}

\end{document}

