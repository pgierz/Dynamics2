\documentclass[a4paper,12pt]{article}
\usepackage{fancyhdr,lastpage}
\usepackage{fullpage}
\usepackage{hyperref}
\usepackage{ulem}


\usepackage{natbib}
\usepackage{wasysym}      % \permil
\usepackage{graphicx}
\usepackage{booktabs}
\bibpunct{(}{)}{;}{a}{,}{;}


\pagestyle{fancy}
\lhead{Dynamics 2\\   \quad \\Due date: 26.05.2014}
\rhead{Exercise 3, Summer semester 2014\\Prof. Dr. G. Lohmann\\19.05.2014}


%\bibpunct{(}{)}{;}{a}{,}{;}

%often used special strings in normal Text
\def\delO{$\delta^{18}$O}           % delta-18-O
\def\delD{$\delta$D}                % delta-D
\def\H2Oi{H$_2$$^{18}$O}            % H2-18-O
\def\ra{$\Rightarrow$}              % =>
\def\la{$\Leftarrow$}               % <=
\def\deg{$^{\circ}$}                % Grad
\def\degC{$^{\circ}$C}              % Grad Celsius

%often used special strings in formula environment
\def\fdelO{\delta^{18}\textrm{O}}   % delta-18-O
\def\fdelD{\delta\textrm{D}}        % delta-D
\def\fra{\Rightarrow}               % =>
\def\fla{\Leftarrow}                % <=

\usepackage{listings} % TeX-package for including source-code
\usepackage{xcolor} % colors for highlighting code

% Custom environments, to make it easier to include code-listings from Fortran
% or R. Note that for Fortran it is possible to choose different dialects (we
% use '95' here, which is the highest version that seems to be supported
% natively by the package).


% START: rcode
\lstnewenvironment{rcode}
% start
{\lstset{language=R}}
% end
{}
% END: rcode

% START: fortcode
\lstnewenvironment{fortcode}
% start
{\lstset{language=[95]Fortran}}
% end
{}
% END: fortcode

%often used structures
\def\be{\begin{equation}}
\def\ee{\end{equation}}

\newcommand{\beq}{\begin{eqnarray*}}
\newcommand{\eeq}{\end{eqnarray*}}
\newcommand{\beqn}{\begin{eqnarray}}
\newcommand{\eeqn}{\end{eqnarray}}
\newcommand{\Der}[2]{\frac{\partial #1}{\partial #2}}
\newcommand{\Dtwo}[2]{\frac{\partial^2 #1}{\partial #2^2}}
\newcommand{\Dx}[1]{\frac{\partial #1}{\partial x}}
\newcommand{\Dt}[1]{\frac{\partial #1}{\partial t}}
%\newcommand{\itt}{\footnotesize}
\newcommand{\sff}{\small \sf}
\newcommand{\itt}{\it }
\newcommand{\itc}{\it }
\newcommand{\dx}{\partial_x}
\newcommand{\dy}{\partial_y}
\newcommand{\dz}{\partial_z}
\newcommand{\dt}{\partial_t}
\newcommand{\ad}{{\footnotemark[2]}}


\newcommand{\Eta}{ \boldmath E }
\newcommand{\vbf}{ \boldsymbol }
\newcommand{\fig}{Fig.\,}
\newcommand{\figs}{Figs.\,}
\renewcommand{\beq}{\begin{eqnarray*}}
\renewcommand{\eeq}{\end{eqnarray*}}
\renewcommand{\beqn}{\begin{eqnarray}}
\renewcommand{\eeqn}{\end{eqnarray}}
\newcommand{\OneMUskip}{\mskip 1mu \relax}  
% 18mu=1em f"ur \textfont
\newcommand{\Nat}{{\sf N\OneMUskip}}
\newcommand{\Reals}{{\sf R\OneMUskip}}
\newcommand{\Complex}{{\sf C\OneMUskip}}

%opening
\title{Exercise 3: Solution}
%\author{ Gerrit Lohmann}
%\date{27. June 2011}

\begin{document}
\maketitle
\thispagestyle{fancy}
%
%\vspace{1cm}

 \begin{enumerate}

\item
{\bf Rayleigh-B\'{e}nard convection }
Rayleigh studied the flow occurring in a layer of fluid of uniform
depth $H$, when the temperature difference between the upper- and lower-surfaces
is maintained at a constant value $\Delta T$.
In the case where all motions are parallel to the $x-z$-plane, and no variations
in the direction of the $y$-axis occur, the governing equations can be written as:
\beqn
    \partial_t u + u \partial_x u + w \partial_z u     & = & - \frac{1}{\rho_0} \partial_x p +        \nu \nabla^2 u  \label{eqref:einse}\\
    \partial_t w + u \partial_x w + w \partial_z w   & = &  - \frac{1}{\rho_0} \partial_z p +        \nu \nabla^2 w + g (1- \alpha (T-T_0))
        \label{eqref:zwei}\\
    \partial_t T + u \partial_x T + w \partial_z T   &=&  \kappa \nabla^2 T 
    \label{eqref:temp}\\
    \partial_x u + \partial_z w   &= &0\label{eqref:kont}
\eeqn
where $w$ and $u$ are the vertical and horizontal components of the velocity,
respectively. 
Furthermore, 
$ \nu = \eta/\rho_0 , \, \kappa = \lambda/(\rho_0 C_v) $ 
the momentum diffusivity (kinematic viscosity) and thermal diffusivity, respectively.

%\begin{verbatim}

The R codes are 
\begin{verbatim}
rb_plot_functions.R; 
rb_functions.R; 
rayleigh-benard.R
\end{verbatim}
In R, you shall make
\begin{rcode}
source("rayleigh-benard.R")
\end{rcode}
and the model run will create a directory out and writes output: 
\begin{rcode}
#Output Parameter
N_out = 50;
out_dir = "./out/";	#Output directory
\end{rcode}


Tasks:
 \begin{enumerate}
 \item
 Write down the dimensionless parameters and their relation to
 characteristic length and time scales charachterizing the flow!

\textbf{Solution} The Dimensionless parameters are:

Prandl Number $\frac{c_p \mu}{k}$
Rayleigh Number $R_a = \frac{g \cdot \beta \cdot \Delta T \cdot
  L^{3}}{\nu \k}$

 \item
Vary the Rayleigh and the Prandtl number by 
$R_a= 20000, 40000, 60000$  and
$Pr= 0.5, 1, 1.5, 5, 10 $ and 
describe the dynamics (words, figures) !


\textbf{Figures for various combinations of Ra and Pr are presented in the
course script.}

\end{enumerate}

\end{document}