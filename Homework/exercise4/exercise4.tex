\documentclass[a4paper,12pt]{article}
\usepackage{fancyhdr,lastpage}
\usepackage{fullpage}
\usepackage{hyperref}
\usepackage{ulem}
\pagestyle{fancy}
\lhead{Dynamics 2\\Lecturer: Prof. Dr. G. Lohmann\\Due date: 16.06.2014}
\rhead{Exercise 4, Summer semester 2014\\Paul Gierz\\02.06.2014}

%opening
\title{Exercise 4}
%\author{ Gerrit Lohmann}
\date{2. June 2014}

\begin{document}

\maketitle
\thispagestyle{fancy}

The aim of this exercise is to analyse a proxy time series by
computing the field correlations with gridded climate data sets (land
surface temperature, sea surface temperature (SST), sea level pressure
(SLP) and precipitation). 

\begin{enumerate}

\item Plot the complete time series and the time series during the
  instrumental period (1900 to 2002). You can choose a time window of
  the time series in the shiny interface. Choose between either the
  ``Grape Harvest Day'' index, or the ``North Atlantic Oscillation''
\\


\item Correlate the proxy time series with all seasons of the data sets (DJF, MAM, JJA, SON) and the annual mean. You can either show the global map or an interesting area of the map ( e.g. if there is only a local correlation).\\


\item Describe the correlation maps without interpretation. You can describe all seasons of one climate variable at once.  
\\
\item Carefully discuss the results and speculate about the climate dependence of the proxy time series.  
\\
\item Examine the Composite Maps. They show the amount of change in
  the units of the examined variable. Describe what you see for each season and what
  it means.

\end{enumerate}

\textbf{Important hints:}
\begin{itemize}
\item Download the updated palaeolibrary and the shiny interface.
\item  In most cases, it is good to plot all four seasons and the
  average mean correlation with the same colour range (zlim). It is
  recommended to take the maximum amount of correlation of all seasons
  as the limit value. 
\item A readme is provided to help with the installation of the PalLib
  shiny interface.

\end{itemize}
\newpage
The points are given as follows for each variable:

\begin{table}[h]
\centering
\begin{tabular}{lr}
	Plot of the results & \textit{ 2 points}\\
	Description of the results& \textit{ 4 points}\\
	Interpretation &\textit{ 4 points}

\end{tabular}
\caption{Points for this exercise}
\label{tab:}
\end{table}

\vfill
\underline{Notes on submission form of the exercises:}
 \textit{Two students work together in one group. Each group has to submit only one solution. The answers to the questions shall be send to paul.gierz@awi.de.}

\end{document}

