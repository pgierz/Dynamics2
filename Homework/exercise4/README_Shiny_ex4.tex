\documentclass[a4paper,12pt]{article}
\usepackage{fancyhdr,lastpage}
\usepackage{fullpage}
\usepackage{hyperref}
\usepackage{ulem}

%R part
\usepackage{listings}
\usepackage{color}
\lstset{ %
  language=R,                     % the language of the code
  basicstyle=\footnotesize,       % the size of the fonts that are used for the code
  numbers=left,                   % where to put the line-numbers
  numberstyle=\tiny\color{gray},  % the style that is used for the line-numbers
  stepnumber=1,                   % the step between two line-numbers. If it's 1, each line
                                  % will be numbered
  numbersep=5pt,                  % how far the line-numbers are from the code
  backgroundcolor=\color{white},  % choose the background color. You must add \usepackage{color}
  showspaces=false,               % show spaces adding particular underscores
  showstringspaces=false,         % underline spaces within strings
  showtabs=false,                 % show tabs within strings adding particular underscores
  frame=single,                   % adds a frame around the code
  rulecolor=\color{black},        % if not set, the frame-color may be changed on line-breaks within not-black text (e.g. commens (green here))
  tabsize=2,                      % sets default tabsize to 2 spaces
  captionpos=b,                   % sets the caption-position to bottom
  breaklines=true,                % sets automatic line breaking
  breakatwhitespace=false,        % sets if automatic breaks should only happen at whitespace
  title=\lstname,                 % show the filename of files included with \lstinputlisting;
                                  % also try caption instead of title
  keywordstyle=\color{blue},      % keyword style
  commentstyle=\color{dkgreen},   % comment style
  stringstyle=\color{mauve},      % string literal style
  escapeinside={\%*}{*)},         % if you want to add a comment within your code
  morekeywords={*,...}            % if you want to add more keywords to the set
} 
\usepackage{fancyvrb}
\usepackage[usenames,dvipsnames]{xcolor}
%page setup
\pagestyle{fancy}
\lhead{Dynamics 2\\Lecturer: Prof. Dr. G. Lohmann}
\rhead{Exercise 4, SS 2014\\Paul Gierz}

%opening
\title{Exercise 4}
\author{--Addendum--}
\date{Installing and using R Shiny for PalLib}

\begin{document}

\maketitle
\thispagestyle{fancy}
\section*{Installing \texttt{shiny}}
In order to make sure everything works correctly, be aware that we have tested for \texttt{R 3.1}. Upon opening R, type the following:

\begin{verbatim}
    
> version

\end{verbatim}
R will then output something similar to this:
\begin{verbatim}
                _                           
platform       x86_64-apple-darwin13.1.0   
arch           x86_64                      
os             darwin13.1.0                
system         x86_64, darwin13.1.0        
status
\end{verbatim}
\begin{Verbatim}[formatcom=\color{OliveGreen}]                                     
major          3                           
minor          1.0                         
\end{Verbatim}
\begin{verbatim}
year           2014                        
month          04                          
day            10                          
svn rev        65387                       
language       R
\end{verbatim}
\begin{Verbatim}[formatcom=\color{OliveGreen}]                            
version.string R version 3.1.0 (2014-04-10)
nickname       Spring Dance
\end{Verbatim}

The important things to check for are the \textcolor{OliveGreen}{\texttt{version.string}} or the \textcolor{OliveGreen}{\texttt{major}} and \textcolor{OliveGreen}{\texttt{minor}} release numbers. If they are incorrect, please try to update. Next, install the \texttt{shiny} library via:
\begin{verbatim}
> install.packages("shiny")
\end{verbatim}
You will need to select a \texttt{CRAN} mirror. Choosing \texttt{0-Cloud}, \texttt{Germany (Berlin)}, \texttt{Germany (Bonn)}, or \texttt{Germany (Goettingen)} will all work. I would recommend against choosing a server in another country as download times may increase. If there are any errors (usually from unresolved dependencies), install the required packages first before re-attempting to install \texttt{shiny}. R will tell you which package it cannot find. A successful installation should look something like this:

\begin{verbatim}
The downloaded binary packages are in
	/var/folders/g9/rnnpj68x1zx4ttywdy67gnpw0000gp/T//RtmpExWkgS/downloaded_packages
\end{verbatim}

Bear in mind that the directory will not be identical! You can test if \texttt{shiny} was installed correctly by giving:
\begin{verbatim}
> library(shiny)
>
\end{verbatim}

This loads the \texttt{shiny} library, which you will need to do any time you wish to run a \texttt{shiny} program. If there are no errors, \texttt{shiny} is installed correctly.

\section*{Starting the GUI}

To begin, make sure you have all the relevant files. In R, navigate to the correct directory and give a \texttt{dir()}.
\begin{verbatim}
> setwd("/Users/Gierz/Documents/Uni/Doktor/Teaching/Dynamics2/PalLib/shiny-project/")
> # Your's will be different!
> dir()
[1] "paleoLibrary"                "run shiny app (windows).cmd"
[3] "server.R"                    "standard.shiny.R"           
[5] "ui.R"            
\end{verbatim}
Compare your \texttt{dir()} command, and make sure you have all of the \texttt{.R} files. If everything is there, you can proceed with:
\begin{verbatim}
> library(shiny)
> runApp(".")

    Listening on http://127.0.0.1:3599
    
\end{verbatim}

The \texttt{runApp(''.'')} will probably fail, as you need to
install some packages first. One is particularly tricky to install,
\texttt{clim.pact}. To do so, install the package \texttt{devtools}
first, load it (with the library(devtools) command), and download the
\texttt{clim.pact\_2.3-10.tar} file from the website. You can install
devtools by using the \texttt{install\_local(path-to-tar)} command. You
may also need to install other packages. Look at the error messages R
sends when trying the command \texttt{runApp(''.'')}. 

\vspace{1cm}
\section*{Using the GUI}
In the file \texttt{standard.shiny.R} you will need to adapt the data
path to where you have put the unpacked \texttt{PalLib.tar.gz}
folder. Save this file before proceeding with runApp.

\texttt{R} will launch a web browser. You can select a timeseries, a
field variable, and a time frame. In the correlation tab, you can
compute the correlation between the timeseries and the selected field
variable. In the composite map tab, you can compute the composite $+-$
map.

Other tips:
\begin{itemize}
\item Sometimes, the program will pause, especially when making the
  spatial plots. Check the R console and push enter (return) if necessary.
\item You can select to show a plot, a histogram, and a summary of the
  file in the Choose Field
\item You can also select which composite to show in the Composite
  Analysis tab.
\item To download the tar.gz file, use ftp:

\begin{itemize}
\item see the ftp-howto on the website.
\item the file is in \texttt{incoming/pgierz/PalLib\_new.tar.gz}; if it has been
  deleted, email me and I will upload it again.
\end{itemize}

\end{itemize}

\vfill
\underline{Notes on submission form of the exercises:}
 \textit{Students may work together in groups, but each student is responsible for her/his own solutions. The answers to the questions shall be send to paul.gierz@awi.de.
}

\end{document}