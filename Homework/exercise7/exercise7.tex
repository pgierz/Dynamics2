\documentclass[a4paper,12pt]{article}
\usepackage{fancyhdr,lastpage}
\usepackage{fullpage}
\usepackage{hyperref}
\usepackage{ulem}
\pagestyle{fancy}
\lhead{Dynamics 2\\Lecturer: Prof. Dr. G. Lohmann\\Due date: 07.06.2014}
\rhead{Exercise 7, Summer semester 2014\\Paul Gierz\\30.06.2014}

%opening
\title{Exercise 7}
%\author{ Gerrit Lohmann}
\date{\today}

\begin{document}

\maketitle
\thispagestyle{fancy}

The aim of this exercise is to understand the propegation of shallow
water waves and examine deep ocean circulation.

There is a R program to calculate both 1D and 2D waves. 

\begin{enumerate}
\item For both 1D and 2D waves:
    \begin{itemize}
        \item Run the program: Which type of waves do you see?
        \item Change the constants of water depth H, gravity g, describe your observations!
        \item Can you roughly estimate the phase speed of the waves?
    \end{itemize}

\item Consider a geostrophic flow $( u,  v) $
\begin{equation}
-f  v = - \frac{1}{\rho_0} \frac{\partial  p}{\partial x} \label{vdyn} \\
f  u = - \frac{1}{\rho_0} \frac{\partial  p}{\partial y} 
\end{equation}
with pressure $ p (x,y,z,t)$. \\
Use the hydrostatic approximation 
\begin{equation}
 \frac{\partial  p}{\partial z} = -g \rho
 \end{equation}
 and equation (\ref{vdyn}) in order to derive the meridional overturning stream function $\Phi(y,z)$ as a fuction of density $\rho$ at the basin boundaries! 
$\Phi$ is defined via 
\begin{equation}
\Phi(y,z) = \int_{0}^{z} \frac{\partial  \Phi}{\partial \tilde z} \,d
\tilde z
\end{equation}
\begin{equation}
\frac{\partial  \Phi}{\partial \tilde z} = \int_{x_e}^{x_w} v(x,y,\tilde z) \, dx  \quad \mbox{(zonally integrated transport)}, 
\end{equation}
where $ x_e $ and $ x_w $ are the eastward and westward boundaries in
the ocean basin (think e.g. of the Atlantic Ocean). Units of $\Phi$
are $m^3 s^{-1}$. At the surface $\Phi(y,0)=0$.
\item
It is observed that water sinks in to the deep ocean in polar regions of the atlantic basin at a rate of 15 $Sv$.
\begin{itemize}
	\item How long would it take to 'fill up' the Atlantic basin?
          (Assume $10^{14}m^{2}$ and $5km$ depth)
	\item Supposing that the local sinking is balanced by
          large-scale upwelling, estimate the strength of this
          upwelling. Express your answer in $m\;y^{-1}.$
\end{itemize}
\end{enumerate}

\vfill
\underline{Notes on submission form of the exercises:}
 \textit{Two students work together in one group. Each group has to submit only one solution. The answers to the questions shall be send to paul.gierz@awi.de.}

\end{document}

