\documentclass[a4paper,12pt]{article}
\usepackage{fancyhdr,lastpage}
\usepackage{fullpage}
\usepackage{hyperref}
\usepackage{ulem}
\usepackage{amsmath}
\pagestyle{fancy}
\lhead{Dynamics 2\\Lecturer: Prof. Dr. G. Lohmann\\Due date: 07.06.2014}
\rhead{Exercise 7, Summer semester 2014\\Paul Gierz\\30.06.2014}

%opening
\title{Exercise 7}
%\author{ Gerrit Lohmann}
\date{\today}

\begin{document}

\maketitle
\thispagestyle{fancy}

The aim of this exercise is to understand the propegation of shallow
water waves and examine deep ocean circulation.

There is a R program to calculate both 1D and 2D waves. 

\begin{enumerate}
\item For both 1D and 2D waves:
    \begin{itemize}
        \item Run the program: Which type of waves do you see?
        \item \textbf{Solution} Both the 1D and 2D waves shown are
          \textit{gravity waves}
        \item Change the constants of water depth H, gravity g,
          describe your observations!
          \item \textbf{Solution} Increasing $g$ causes the waves to
            propegate faster. Increasing $H$ has a similar effect,
            although the peak shapes are altered; the waves do not
            propegate as well in deeper water and are damped more quickly.
        \item Can you roughly estimate the phase speed of the waves?
          \item \textbf{Solution} The phase speed of a wave is given
            mathematically by $c = \sqrt{gH}$
    \end{itemize}

\item Consider a geostrophic flow $( u,  v) $
\begin{equation}
-f  v = - \frac{1}{\rho_0} \frac{\partial  p}{\partial x} \label{vdyn} \\
f  u = - \frac{1}{\rho_0} \frac{\partial  p}{\partial y} 
\end{equation}
with pressure $ p (x,y,z,t)$. \\
Use the hydrostatic approximation 
\begin{equation}
 \frac{\partial  p}{\partial z} = -g \rho
 \end{equation}
 and equation (\ref{vdyn}) in order to derive the meridional overturning stream function $\Phi(y,z)$ as a fuction of density $\rho$ at the basin boundaries! 
$\Phi$ is defined via 
\begin{equation}
\Phi(y,z) = \int_{0}^{z} \frac{\partial  \Phi}{\partial \tilde z} \,d
\tilde z
\end{equation}
\begin{equation}
\frac{\partial  \Phi}{\partial \tilde z} = \int_{x_e}^{x_w} v(x,y,\tilde z) \, dx  \quad \mbox{(zonally integrated transport)}, 
\end{equation}
where $ x_e $ and $ x_w $ are the eastward and westward boundaries in
the ocean basin (think e.g. of the Atlantic Ocean). Units of $\Phi$
are $m^3 s^{-1}$. At the surface $\Phi(y,0)=0$.

\item \textbf{Solution}
  Here, we will find that the overturning strength \textit{only
    depends on the density differences between East and West}. We know
  that at $z=0, \Phi(y,0) = 0$, since at the surface there is no
  overturning. We can start by finding $\Phi(\rho)$, the overturning
  as a function of density.

\begin{flalign}
\frac{\partial \Phi}{\partial z} &= \int_{x_e}^{x_w}v(x,y,z)dx \\
&= \int_{x_e}^{x_w}\frac{1}{\rho_0 g} \frac{\partial p}{\partial x} dx\\
&= \frac{1}{\rho_0 g} p(x_w) - p(x_e)
\end{flalign}

We can now substitute this into the equation for $\Phi(y,z)$.

\begin{flalign}
\Phi(y,z) &= \int_{0}^{z}\frac{\partial \Phi}{\partial z} dz \\
&= \int_{0}^{z} \frac{1}{\rho_0 g} p(x_w) - p(x_e) dz \\
&= \frac{1}{-2 \rho_0 f g \rho} * [p_w(z)^2 - p_e(z)^2 - p_w(0)^2 +
p_e(0)^2 ]
\end{flalign}
We know that pressure at the surface is simply atmospheric standard
pressure, and we can neglect this as small fluctuations of $p_0$
will not impact the overturning strength. We therefore arrive at:
\begin{equation}
\Phi = -\frac{1}{2\rho_0 \rho f g} [p_w(z)^2 - p_e(z)^2]
\end{equation}

You can easily convert pressure to density, since all other factors
remain the same (ie gravity, depth, temperature, and so
on). Therefore, the strength of the overturning depends only on the
pressure (and thereby density) differences between the eastern and
western edge of the basin.

\item
It is observed that water sinks in to the deep ocean in polar regions of the atlantic basin at a rate of 15 $Sv$.
\begin{itemize}
	\item How long would it take to 'fill up' the Atlantic basin?
          (Assume $10^{14}m^{2}$ and $5km$ depth)
          \item \textbf{Solution} $\frac{5 \times 10^{17}}{15 \times
              10^{6}} = 0.33 \times 10^{11}s = 1046.6 yrs$
	\item Supposing that the local sinking is balanced by
          large-scale upwelling, estimate the strength of this
          upwelling. Express your answer in $m\;y^{-1}.$
          \item \textbf{Solution} $\frac{15 \times 10^{6}}{10^{14}} =
            4.7 \frac{m}{y}$
\end{itemize}
\end{enumerate}

\vfill
\underline{Notes on submission form of the exercises:}
 \textit{Two students work together in one group. Each group has to submit only one solution. The answers to the questions shall be send to paul.gierz@awi.de.}

\end{document}

