\documentclass[a4paper,12pt]{article}
\usepackage{fancyhdr,lastpage}
\usepackage{fullpage}
\usepackage{hyperref}
\usepackage{ulem}
\pagestyle{fancy}
\lhead{Dynamics 2\\Lecturer: Prof. Dr. G. Lohmann\\Due date: 23.06.2014}
\rhead{Exercise 5, Summer semester 2014\\Paul Gierz\\16.06.2014}

%opening
\title{Exercise 5 Solution}
%\author{ Gerrit Lohmann}
\date{16. June 2014}

\begin{document}

\maketitle
\thispagestyle{fancy}

In this excercise we will look at Foucault's Pendulum, which is a well known expirement
used to demonstrate the Earth's rotation. One such pendulum is
displayed at the entrance of the physics building at the university.

From the course script:

Let us denote $x$, $y$, $L$, and $\Theta$ as the coordinates of the
pendulum, the length of the string, and the angle of the pendulum bob,
respectively. The string tension due to gravity can be written as:

\begin{equation}
F_{g} = mg \left (\begin{array}{c} sin{\Theta} \\ sin{\Theta} \\
    cos{\Theta} \end{array} \right ) \approx mg \left (\begin{array}{c} x/L \\ y/L \\
    1-z/L \end{array} \right )
\end{equation} 

The horizontal dynamics can be explained via:

\begin{equation}
a_x = 2\Omega sin \varphi v_y - \frac{g}{L}x 
\end{equation}
\begin{equation}
a_y = 2\Omega sin \varphi v_x - \frac{g}{L}y
\end{equation}


\begin{enumerate}

\item Show the analytical solution by introducing the complex number
  $\xi = x + i \cdot y$


\textbf{Solution} 

Multiply 2 by $i$:
$$a_y i = -2 \Omega sin \varphi v_x i - \frac{g}{L} y i$$

Add this into 1 and optain the differential equation for motion in the
horizional plane':

$$\ddot{\xi} = (- 2 i \Omega \varphi) \dot{xi}$$

Thus, we will have two independent solutions. Solving the 2nd order
differential equation introduce a variable $\xi(t) = e^{iat}$ and insert it
above, where a is the solution for the characteristic equation

$$a^{2} + 2i\Omega \varphi a + \frac{g}{L}$$
$$a_{1,2} = -i \Omega sin \varphi \pm i \sqrt{\Omega ^{2} sin^{2}
  \varphi + \frac{g}{L}}$$

So,

$$\xi (t) = x_{1}e^{-ia_{2}t}+x_{2}e^{-ia_{2}t}$$

$x_{1}$ and $x_{2}$ depend on the initial conditions (velocity, and
potision). Solving for $x_{1}$ and $x_{2}$ gives:

$$x_{1} = \frac{1}{2} + \frac{\Omega sin
  \varphi}{\sqrt{\Omega^{2}sin^{2} \varphi + \frac{g}{L}^{2}}}$$

$$x_{2} = \frac{1}{2} - \frac{\Omega sin
  \varphi}{\sqrt{\Omega^{2}sin^{2} \varphi + \frac{g}{L}^{2}}}$$

\item For Foucault's pendulum in Paris: the plane of the pendulum
  rotates clockwise $11^{\circ}$ per hour, making a full circle in
  32.7 hours. What is the period in Bremen?  What is the period at
  your home town?

\textbf{Solution} For Bremen, the period is 30 hours.

\item Using the \texttt{shiny} program provided, compare the
  analytical solution with the numerical solution.

\item Geological evidence suggests that the Earth used to have a
  shorter day in the past, due to the impact forming the moon (angular
  momentum). What is
  the period of the pendulum on an Earth with only a 22 hour day for
  the 3 locations (Paris, Bremen, your home)?

\textbf{Solution} With a shorter day, the period is 28 hours in bremen.
\end{enumerate}


\vfill
\underline{Notes on submission form of the exercises:}
 \textit{Students can work together, but each is required to submit
   his or her own solutions. The answers to the questions shall be send to paul.gierz@awi.de.}

\end{document}

