\documentclass[a4paper,12pt]{article}
\usepackage{fancyhdr,lastpage}
\usepackage{fullpage}
\usepackage{hyperref}
\usepackage{ulem}
\pagestyle{fancy}
\lhead{Dynamics 2\\Lecturer: Prof. Dr. G. Lohmann\\Due date:10.04.2011}
\rhead{Exercise 1, Summer semester 2011\\Prof. Dr. G. Lohmann/Moritz Krieger\\04.04.2011}

%opening
\title{Exercise 1}
%\author{ Gerrit Lohmann}
\date{04. April 2011}

\begin{document}

\maketitle
\thispagestyle{fancy}

\begin{enumerate}

\item Download and install the R-Software.
\url{http://www.r-project.org/}  $\rightarrow$ Download CRAN $\rightarrow$ search a city near you
Choose your system (Windows / Mac / Linux)
Follow the instructions.
\\[1ex]

\item Create a vector $t$  "t$<$-seq(-2*pi,2*pi,by=0.01)"
plot several functions in one window ($\sin(t), \cos(t), \exp(\frac t 5), (\frac t 5)^2, (\frac t 5)^3$). Try some of the plot arguments: Set ylim, label the axes, set a different colour for each function, vary the line width. Save the plot as a figure. \\
For help try "?plot" or "?plot.default"

\textit{ 3 points} \\[2ex]

\item Set up a vector of length 20 and create a vector b with a linear relationship to a (e.g. $a=3b+7$).
Calculate the correlation("cor(a,b)").

\textit{ 1 point} \\[2ex]

\item Set up two random vectors a,b of length 20 and calculate the correlation. Repeat this procedure several times to get a feeling for the correlation coefficient.
Than vary the length of vector a and b (vary the sample number) and discuss how the correlation coefficient changes (e.g. 10,50,100,1000). 

\textit{ 2 points} \\[2ex]

\item Repeat the experiment from task 4 100 times by using a loop.
Create before the loop an empty vector ("cor.val$<$-vector()") and save the correlation of a and b in this vector (e.g. "cor.val[i]$<$-cor(a,b)") for each realisation.
Compute the mean value and plot the histogram of cor.val. 
What happens with the histogram when the length of a and b is varied (e.g. 10,50,100)? Save two different histograms as a figure and explain the difference between them. 

\textit{ 2 points} \\[2ex]


\item Repeat the procedure of task 5. with partly linear dependent vectors:
("a$<$-rnorm(100); b$<$-r*a+rnorm(100)")
Choose one value for r and shortly discuss the mean value and the histogram of cor.val compared to task 5. Save the histogram as a figure.

\textit{ 2 points} \\[2ex]




\begin{verbatim}
	# Important R-commands
	rnorm(N) # create vector with N normal distribution random numbers
	cor(a,b) # calculates the correlation coefficient
	hist(a)  # histogram of vector a 
	mean(a)  # mean value of vector a

	# Helpful introductions to R can be found in e.g.
http://www.stat.cmu.edu/~larry/all-of-statistics/=R/Rintro.pdf
http://cran.r-project.org/doc/manuals/R-intro.pdf
\end{verbatim}
\underline{Notes on submission form of the exercises:}
 \textit{Working in study groups of two persons is encouraged, but each student should programme his/her own code.
The programme code and the answers to the questions including the figures shall be send as one document to \url{kriegerm@uni-bremen.de} (until Sunday!).
}

\end{enumerate}




\end{document}

